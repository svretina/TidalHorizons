\documentclass[reprint, prd, nofootinbib, superscriptaddress, floatfix]{revtex4-2}  % chktex 8

\usepackage{graphicx}% Include figure files
\usepackage{dcolumn}% Align table columns on decimal point
\usepackage{bm}% bold math
\usepackage[linktocpage,colorlinks,pdfusetitle,allcolors=blue]{hyperref}
\usepackage{amsmath, amssymb}
\usepackage{siunitx}
\usepackage{cancel}


\renewcommand{\d}[1]{\ensuremath{\operatorname{d}\!{#1}}}
\renewcommand{\vec}[1]{\boldsymbol{#1}}

\newlength\figureheight
\newlength\figurewidth
\setlength\figureheight{\linewidth}
\setlength\figurewidth{\linewidth}


\newcommand{\eq}[1]{Eq.\ \eqref{#1}}
\newcommand{\fig}[1]{Fig.\ \ref{#1}}
\newcommand{\SV}[1]{{\color{red}{[[Stamatis:: #1]]}}}
\newcommand{\BK}[1]{{\color{green}{[[Badri:: #1]]}}}
\newcommand{\GB}[1]{{\color{blue}{[[Erik:: #1]]}}}
\newcommand{\Mdot}{\langle \dot{M} \rangle}
\newcommand{\Jdot}{\langle \dot{J} \rangle}




\begin{document}

\title{Tidal Horizons}

\author{Stamatis Vretinaris}
\email{svretina@physics.auth.gr}
\affiliation{
  Institute for Mathematics, Astrophysics and Particle Physics, Radboud University, Heyendaalseweg 135, 6525 AJ Nijmegen, The Netherlands
}

\author{Badri Krishnan}
\affiliation{
  Institute for Mathematics, Astrophysics and Particle Physics, Radboud University, Heyendaalseweg 135, 6525 AJ Nijmegen, The Netherlands
}

\author{Erik Schnetter}
\affiliation{
  Perimeter Institute for Theoretical Physics, Waterloo, Ontario,
  N2L 2Y5, Canada}
\affiliation{
Department of Physics & Astronomy, University of Waterloo,
Waterloo, ON N2L 3G1, Canada}
\affiliation{
Center for Computation & Technology, Louisiana State
University, Baton Rouge, LA 70803, USA
}

\date{\today}

\begin{abstract}
abstract\end{abstract}

\maketitle
\section{Introduction}
Horizon data is generated using the Kerr Shild data for a single black hole of mass 1 and spin 0.

We use the QuasiLocalMeasures thorn to extract information on the horizon such as the $\Psi_{2}$ scalar.
For a Schwarzschild black hole this is
\begin{equation}
  \label{eq:psi2-schwarzschild}
  \Psi_{2} = -\frac{M}{r^{3}}
\end{equation}


We expand the $\Psi_{2}$ scalar into spherical harmonics
\begin{equation}
  \label{eq:expansion}
  \Psi_{2} = \sum_{l=0}^{\infty}\sum_{m=-l}^{l} C_{l}^{m}Y_{l}^{m}(\vartheta, \phi)
\end{equation}
where
\begin{equation}
  \label{eq:spherical-harmonics}
  Y_{l}^{m}= (-1)^{m} \sqrt{\frac{(2l+1)}{4\pi} \frac{(l-m)!}{(l+m)!}} P_{l}^{m} (\cos \vartheta) e^{im\phi}
\end{equation}
where $P_{l}^{m}$ are the associated Legendre polynomials.

as a test we decompose \eq{eq:psi2-schwarzschild}
\begin{equation}
  \label{eq:psi2-expansion}
  C_{l}^{m} = \int_{\Omega} \Psi_{2} Y_{l}^{m}(\vartheta, \phi) d\Omega = \Psi_{2} \int_{0}^{2\pi} d\phi
\int_{0}^{\pi }Y_{l}^{m}\sin\vartheta d\vartheta
\end{equation}
all coefficients $C_{l}^{m}$ except the $C_{0}^{0}$
\begin{equation}
  C_{0}^{0} = \Psi_{2} \int_{0}^{2\pi} d\phi \int_{0}^{\pi }Y_{0}^{0}\sin\vartheta d\vartheta = 2\sqrt{\pi} \Psi_{2}
\end{equation}
which evaluates to $C_{0}^{0}=-0.443114$ for a Schwarzschild black hole of mass 1.

\section{Code description}
We use the \texttt{Carpet} infrastructure to produce initial data for a single Schwarzschild black hole by using the Kerr Schild data provided by the \texttt{Exact} thorn. The Weyl scalars are extracted by the \texttt{QuasiLocalMeasures} thorn and are imported through \texttt{kuibit}.

For the spherical decompositions we use the \texttt{pyshtools} library.
We interpolate the horizon data onto a Gauss-Legendre grid and decompose it.
The numerical value of $C_{0}^{0}$ we get is -0.44310608.

The accuracy of the interpolation is low and erros are introduced at this step.


Next steps:\\
- two schwarzschild black holes\\
- one schwarzschild one kerr\\
- two kerr\\

run for various values of\\
- distance, spin, mass ratio


\end{document}
